%%%%%%%%%%%%%%%%%%%%%%%%%%%%%%%%%%%%%%%%%
% University Assignment Title Page 
% LaTeX Template
% Version 1.0 (27/12/12)
%
% This template has been downloaded from:
% http://www.LaTeXTemplates.com
%
% Original author:
% WikiBooks (http://en.wikibooks.org/wiki/LaTeX/Title Creation)
%
% License:
% CC BY-NC-SA 3.0 (http://creativecommons.org/licenses/by-nc-sa/3.0/)
% 
% Instructions for using this template:
% This title page is capable of being compiled as is. This is not useful for 
% including it in another document. To do this, you have two options: 
%
% 1) Copy/paste everything between \begin{document} and \end{document} 
% starting at \begin{titlepage} and paste this into another LaTeX file where you 
% want your title page.
% OR
% 2) Remove everything outside the \begin{titlepage} and \end{titlepage} and 
% move this file to the same directory as the LaTeX file you wish to add it to. 
% Then add \input{./title page_1.tex} to your LaTeX file where you want your
% title page.
%
%%%%%%%%%%%%%%%%%%%%%%%%%%%%%%%%%%%%%%%%%

%----------------------------------------------------------------------------------------
%	PACKAGES AND OTHER DOCUMENT CONFIGURATIONS
%----------------------------------------------------------------------------------------

\documentclass[12pt]{article}
\usepackage{graphicx}
\usepackage[utf8]{inputenc}  
\usepackage[T1]{fontenc} 
\usepackage[top=1cm,bottom=1cm,left=0.5cm,right=1.5cm,asymmetric]{geometry}
\usepackage{amsfonts}
\usepackage{graphicx}
\usepackage{amsmath}
\usepackage{caption}
\usepackage{subcaption}
\usepackage{float}
\usepackage{subfig}
\usepackage{fancyhdr}
\pagestyle{fancy}
\renewcommand{\footrulewidth}{1pt}
\fancyhead[R]{\textit{Master MVA : Graphs in Machine Learning }}
\fancyfoot[L]{\textit{}}
%\usepackage{unicode-math}
%\setmathfont{XITS Math}
%\setmathfont[version=setB,StylisticSet=1]{XITS Math}
\usepackage{array,multirow,makecell}
\setcellgapes{1pt}
\makegapedcells
\newcolumntype{R}[1]{>{\raggedleft\arraybackslash }b{#1}}
\newcolumntype{L}[1]{>{\raggedright\arraybackslash }b{#1}}
\newcolumntype{C}[1]{>{\centering\arraybackslash }b{#1}}

\pagestyle{fancy}
\renewcommand{\footrulewidth}{1pt}
\fancyfoot[L]{\textit{}}
\newcommand{\cond}{(x_i|x_{\pi_i})}

%\usepackage{caption}
%\usepackage{subcaption}


%\usepackage{unicode-math}
%\setmathfont{XITS Math}
%\setmathfont[version=setB,StylisticSet=1]{XITS Math}


%\geometry{hmargin=1.5cm,vmargin=2cm}   

\geometry{hmargin=2.5cm,vmargin=2cm}   
\begin{document}

\section*{Graphs in Machine Learning : TP2}
\section*{Sammy Khalife}
\subsubsection*{24/02/2015}

class 4 slide 36 - 41 for HFS and CHFS
~\\
sogma term in exponential be careful : debug
0.025
2~\\
~\\
openCV~\\
cap=cv.VideoCapture('.avi')~\\
cap.read~\\
section 2 = use open CV~\\
2 steps :~\\
-cascade detector : c.detect : list of boxes~\\
-treat data smoothing 96 *96 image into a feature vector~\\
Hard part : ~\\
- build similarity metrics : compare 2 faces with 2 vectors : compute weights to create invariance~\\
-labels = my face and face of other guys.~\\
-implement recognition part : propagate labels using HFS~\\
-centroid method to add node to a graph~\\
Difficulty of the third part : ~\\
keep track which of which node is playing centroid role for nodes~\\
Make link between nodes and centroids
-nodesToCentroids.map[1:t];~\\
-centroidsToNodes.map[]~\\
-at the end compute Laplacian of the centroids~\\
-HFS on this Laplacian~\\
-How to update the structure? : accumulate a bunch of nodes, 200, 300 steps : does not work online : small cost at each step. instead of adding at each step, find 2 closest centroids in the approximations, do not remove the labeled nodes~\\
-keep a taboo list~\\
-cadd~\\
-crep~\\
~\\
find two close centroids, merge labels associated to the 2 close centroids, and new labels associated to the vacant centroid
\end{document}